\documentclass[a4paper, 12pt]{article}
\usepackage[margin=0.8in]{geometry}
\usepackage[utf8]{inputenc}
\usepackage{amsmath, amsfonts}
\usepackage[T1]{fontenc}
\usepackage{xfrac}
\usepackage{amsthm}

\usepackage[english, greek]{babel}
\usepackage[T1, T2A]{fontenc}

\newtheoremstyle{deault_theorem_style}
{}                % Space above
{2}                % Space below
{\upshape}        % Theorem body font % (default is "\upshape")
{}                % Indent amount
{\bfseries}       % Theorem head font % (default is \mdseries)
{}               % Punctuation after theorem head % default: no punctuation
{ }               % Space after theorem head
{}                % Theorem head spec

\theoremstyle{deault_theorem_style}


\newtheorem{protodefn}{Definition}[section]
% \newenvironment{defn}{\begin{protodefn}}{\end{protodefn}}
\newenvironment{defn}
   {\colorlet{shadecolor}{black!6}\begin{shaded}\begin{protodefn}}
   {\end{protodefn}\end{shaded}}
   
\newtheorem{sqstn}[]{\Large{Question}}

\newtheorem{protoinote}[protodefn]{Important Note}
\newenvironment{inote}{\begin{protoinote}}{\end{protoinote}}

\newcommand\Linepage[1][0.3in]{% Change to suit
  \vbox to \dimexpr\textheight-\pagetotal-#1\relax {% Let TeX do the work...
    \leaders\hbox to \linewidth{\rule{0pt}{#1}\hrulefill}\vfil
  }%
}



\title{Mathematics Pancyprian Examination Past Papers}
\author{Andreas Mitillos}
\date{July 2022}

\begin{document}

\selectlanguage{english}
\begin{sqstn}{\ }
    \flushleft\begin{enumerate}
        \item Express $7\sin (2\theta) - 2 \cos (2 \theta)$ in the form $R\sin(2 \theta - \alpha)$, where $R$ and $\alpha$ are constants, $R>0$ and $0< \alpha < 90 ^o$. Give the exact value of $R$ and give the value of $\alpha$ to $2$ decimal places.

        \item Hence solve, for $0 \leq \theta < 90^o$, the equation 
        \[ 7 \sin (2 \theta ) - 2 \cos (2 \theta) = 4 \]
        giving your answers in degrees to one decimal place. 

        \item Express $28 \sin (\theta)\cos(\theta) + 8 \sin ^2 (\theta)$ in the form $a \sin (2\theta) + b \cos (2\theta) + c$, where $a, \ b$ and $c$ are constants to be found. 

        \item Use your answers to part (a) and part (c) to deduce the exact maximum value of $28 \sin (\theta) \cos (\theta) + 8 \sin ^2 (\theta)$
    \end{enumerate}
    \flushright \textbf{(20)}
    \Linepage
\end{sqstn}

\newpage
\Linepage

\hfill
\newpage
\begin{sqstn}{\ }
    \flushleft Given that 
    \[ \frac{3x^2 + 4x - 7}{(x+1)(x-3)} \equiv A + \frac{B}{x+1} + \frac{C}{x-3} \]
    \begin{enumerate}
        \item find the values of the constants $A$, $B$ and $C$.
        \item Hence, or otherwise, find the series expansion of 
        \[ \frac{3x^2 + 4x - 7}{(x+1)(x-3)} \ \ \ \ |x| < 1 \]
        in ascending powers of $x$, up to and including the term in $x^2$.
        \flushleft Give each coefficient as a simplified fraction.
    \end{enumerate}
    \flushright \textbf{(10)}
    \Linepage
\end{sqstn}

\newpage
\Linepage

\hfill
\newpage
\begin{sqstn}{\ }
    \flushleft The function $f$ is defined by 
    \[ f(x) = 2x^2 + 3kx + k^2 \ \ \ \ x\in \mathbb{R}, \ -4k \leq x \leq 0 \]
    where $k$ is a positive constant.
    \begin{enumerate}
        \item Find, in terms of $k$, the range of $f$.
    \end{enumerate}
    The function $g$ is defined by 
    \[ g(x) = 2k - 3x \ \ \ \ x \in \mathbb{R} \]
    Given that $gf(-2) = -12$
    \begin{enumerate}
        \item find the possible values of $k$.
    \end{enumerate}
    \flushright \textbf{(15)}
    \Linepage
\end{sqstn}

\hfill

\newpage
\Linepage
\newpage

\newpage
\begin{sqstn}{\ }
    \flushleft The curve $C$ has equation
    \[ 81y^3 + 64x^2y + 256x = 0 \]
    \begin{enumerate}
        \item Find $\frac{\text{dy}}{\text{dx}}$ in terms of $x$ and $y$.
        \item Hence find the coordinates of the points on $C$ where $\frac{\text{dy}}{\text{dx}}=0$
    \end{enumerate}
    \flushright \textbf{(5)}

    \Linepage
\end{sqstn}

\hfill

\newpage

\Linepage

\newpage

\begin{sqstn}{\ }
    \flushleft The angle $x$ and the angle $y$ are such that 
    \[ \tan (x) = m \ \text{ and } \ 4\tan y = 8m +5 \]
    where $m$ is a constant.
    
    \flushleft Given that $16 \sec^2 x + 16 \sec^2 y = 537$
    
    \begin{enumerate}
        \item find the two possible values of $m$
    \end{enumerate}  

    \flushleft Given that the angle $x$ and the angle $y$ are acute, find the exact value of 
    
    \begin{enumerate}
        \item $\sin (x)$
        \item $\cot (y)$
    \end{enumerate}
    \flushright \textbf{(20)}

    \Linepage	
\end{sqstn}

\newpage
\Linepage


\end{document}
